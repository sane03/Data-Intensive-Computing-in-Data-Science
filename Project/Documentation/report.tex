%% bare_jrnl.tex
%% V1.4b
%% 2015/08/26
%% by Michael Shell
%% see http://www.michaelshell.org/
%% for current contact information.
%%
%% This is a skeleton file demonstrating the use of IEEEtran.cls
%% (requires IEEEtran.cls version 1.8b or later) with an IEEE
%% journal paper.
%%
%% Support sites:
%% http://www.michaelshell.org/tex/ieeetran/
%% http://www.ctan.org/pkg/ieeetran
%% and
%% http://www.ieee.org/

%%*************************************************************************
%% Legal Notice:
%% This code is offered as-is without any warranty either expressed or
%% implied; without even the implied warranty of MERCHANTABILITY or
%% FITNESS FOR A PARTICULAR PURPOSE! 
%% User assumes all risk.
%% In no event shall the IEEE or any contributor to this code be liable for
%% any damages or losses, including, but not limited to, incidental,
%% consequential, or any other damages, resulting from the use or misuse
%% of any information contained here.
%%
%% All comments are the opinions of their respective authors and are not
%% necessarily endorsed by the IEEE.
%%
%% This work is distributed under the LaTeX Project Public License (LPPL)
%% ( http://www.latex-project.org/ ) version 1.3, and may be freely used,
%% distributed and modified. A copy of the LPPL, version 1.3, is included
%% in the base LaTeX documentation of all distributions of LaTeX released
%% 2003/12/01 or later.
%% Retain all contribution notices and credits.
%% ** Modified files should be clearly indicated as such, including  **
%% ** renaming them and changing author support contact information. **
%%*************************************************************************


% *** Authors should verify (and, if needed, correct) their LaTeX system  ***
% *** with the testflow diagnostic prior to trusting their LaTeX platform ***
% *** with production work. The IEEE's font choices and paper sizes can   ***
% *** trigger bugs that do not appear when using other class files.       ***                          ***
% The testflow support page is at:
% http://www.michaelshell.org/tex/testflow/



\documentclass[journal]{IEEEtran}
%
% If IEEEtran.cls has not been installed into the LaTeX system files,
% manually specify the path to it like:
% \documentclass[journal]{../sty/IEEEtran}





% Some very useful LaTeX packages include:
% (uncomment the ones you want to load)


% *** MISC UTILITY PACKAGES ***
%
%\usepackage{ifpdf}
% Heiko Oberdiek's ifpdf.sty is very useful if you need conditional
% compilation based on whether the output is pdf or dvi.
% usage:
% \ifpdf
%   % pdf code
% \else
%   % dvi code
% \fi
% The latest version of ifpdf.sty can be obtained from:
% http://www.ctan.org/pkg/ifpdf
% Also, note that IEEEtran.cls V1.7 and later provides a builtin
% \ifCLASSINFOpdf conditional that works the same way.
% When switching from latex to pdflatex and vice-versa, the compiler may
% have to be run twice to clear warning/error messages.






% *** CITATION PACKAGES ***
%
%\usepackage{cite}
% cite.sty was written by Donald Arseneau
% V1.6 and later of IEEEtran pre-defines the format of the cite.sty package
% \cite{} output to follow that of the IEEE. Loading the cite package will
% result in citation numbers being automatically sorted and properly
% "compressed/ranged". e.g., [1], [9], [2], [7], [5], [6] without using
% cite.sty will become [1], [2], [5]--[7], [9] using cite.sty. cite.sty's
% \cite will automatically add leading space, if needed. Use cite.sty's
% noadjust option (cite.sty V3.8 and later) if you want to turn this off
% such as if a citation ever needs to be enclosed in parenthesis.
% cite.sty is already installed on most LaTeX systems. Be sure and use
% version 5.0 (2009-03-20) and later if using hyperref.sty.
% The latest version can be obtained at:
% http://www.ctan.org/pkg/cite
% The documentation is contained in the cite.sty file itself.






% *** GRAPHICS RELATED PACKAGES ***
%
\ifCLASSINFOpdf
  % \usepackage[pdftex]{graphicx}
  % declare the path(s) where your graphic files are
  % \graphicspath{{../pdf/}{../jpeg/}}
  % and their extensions so you won't have to specify these with
  % every instance of \includegraphics
  % \DeclareGraphicsExtensions{.pdf,.jpeg,.png}
\else
  % or other class option (dvipsone, dvipdf, if not using dvips). graphicx
  % will default to the driver specified in the system graphics.cfg if no
  % driver is specified.
  % \usepackage[dvips]{graphicx}
  % declare the path(s) where your graphic files are
  % \graphicspath{{../eps/}}
  % and their extensions so you won't have to specify these with
  % every instance of \includegraphics
  % \DeclareGraphicsExtensions{.eps}
\fi
% graphicx was written by David Carlisle and Sebastian Rahtz. It is
% required if you want graphics, photos, etc. graphicx.sty is already
% installed on most LaTeX systems. The latest version and documentation
% can be obtained at: 
% http://www.ctan.org/pkg/graphicx
% Another good source of documentation is "Using Imported Graphics in
% LaTeX2e" by Keith Reckdahl which can be found at:
% http://www.ctan.org/pkg/epslatex
%
% latex, and pdflatex in dvi mode, support graphics in encapsulated
% postscript (.eps) format. pdflatex in pdf mode supports graphics
% in .pdf, .jpeg, .png and .mps (metapost) formats. Users should ensure
% that all non-photo figures use a vector format (.eps, .pdf, .mps) and
% not a bitmapped formats (.jpeg, .png). The IEEE frowns on bitmapped formats
% which can result in "jaggedy"/blurry rendering of lines and letters as
% well as large increases in file sizes.
%
% You can find documentation about the pdfTeX application at:
% http://www.tug.org/applications/pdftex





% *** MATH PACKAGES ***
%
%\usepackage{amsmath}
% A popular package from the American Mathematical Society that provides
% many useful and powerful commands for dealing with mathematics.
%
% Note that the amsmath package sets \interdisplaylinepenalty to 10000
% thus preventing page breaks from occurring within multiline equations. Use:
%\interdisplaylinepenalty=2500
% after loading amsmath to restore such page breaks as IEEEtran.cls normally
% does. amsmath.sty is already installed on most LaTeX systems. The latest
% version and documentation can be obtained at:
% http://www.ctan.org/pkg/amsmath





% *** SPECIALIZED LIST PACKAGES ***
%
%\usepackage{algorithmic}
% algorithmic.sty was written by Peter Williams and Rogerio Brito.
% This package provides an algorithmic environment fo describing algorithms.
% You can use the algorithmic environment in-text or within a figure
% environment to provide for a floating algorithm. Do NOT use the algorithm
% floating environment provided by algorithm.sty (by the same authors) or
% algorithm2e.sty (by Christophe Fiorio) as the IEEE does not use dedicated
% algorithm float types and packages that provide these will not provide
% correct IEEE style captions. The latest version and documentation of
% algorithmic.sty can be obtained at:
% http://www.ctan.org/pkg/algorithms
% Also of interest may be the (relatively newer and more customizable)
% algorithmicx.sty package by Szasz Janos:
% http://www.ctan.org/pkg/algorithmicx




% *** ALIGNMENT PACKAGES ***
%
%\usepackage{array}
% Frank Mittelbach's and David Carlisle's array.sty patches and improves
% the standard LaTeX2e array and tabular environments to provide better
% appearance and additional user controls. As the default LaTeX2e table
% generation code is lacking to the point of almost being broken with
% respect to the quality of the end results, all users are strongly
% advised to use an enhanced (at the very least that provided by array.sty)
% set of table tools. array.sty is already installed on most systems. The
% latest version and documentation can be obtained at:
% http://www.ctan.org/pkg/array


% IEEEtran contains the IEEEeqnarray family of commands that can be used to
% generate multiline equations as well as matrices, tables, etc., of high
% quality.




% *** SUBFIGURE PACKAGES ***
%\ifCLASSOPTIONcompsoc
%  \usepackage[caption=false,font=normalsize,labelfont=sf,textfont=sf]{subfig}
%\else
%  \usepackage[caption=false,font=footnotesize]{subfig}
%\fi
% subfig.sty, written by Steven Douglas Cochran, is the modern replacement
% for subfigure.sty, the latter of which is no longer maintained and is
% incompatible with some LaTeX packages including fixltx2e. However,
% subfig.sty requires and automatically loads Axel Sommerfeldt's caption.sty
% which will override IEEEtran.cls' handling of captions and this will result
% in non-IEEE style figure/table captions. To prevent this problem, be sure
% and invoke subfig.sty's "caption=false" package option (available since
% subfig.sty version 1.3, 2005/06/28) as this is will preserve IEEEtran.cls
% handling of captions.
% Note that the Computer Society format requires a larger sans serif font
% than the serif footnote size font used in traditional IEEE formatting
% and thus the need to invoke different subfig.sty package options depending
% on whether compsoc mode has been enabled.
%
% The latest version and documentation of subfig.sty can be obtained at:
% http://www.ctan.org/pkg/subfig




% *** FLOAT PACKAGES ***
%
%\usepackage{fixltx2e}
% fixltx2e, the successor to the earlier fix2col.sty, was written by
% Frank Mittelbach and David Carlisle. This package corrects a few problems
% in the LaTeX2e kernel, the most notable of which is that in current
% LaTeX2e releases, the ordering of single and double column floats is not
% guaranteed to be preserved. Thus, an unpatched LaTeX2e can allow a
% single column figure to be placed prior to an earlier double column
% figure.
% Be aware that LaTeX2e kernels dated 2015 and later have fixltx2e.sty's
% corrections already built into the system in which case a warning will
% be issued if an attempt is made to load fixltx2e.sty as it is no longer
% needed.
% The latest version and documentation can be found at:
% http://www.ctan.org/pkg/fixltx2e


%\usepackage{stfloats}
% stfloats.sty was written by Sigitas Tolusis. This package gives LaTeX2e
% the ability to do double column floats at the bottom of the page as well
% as the top. (e.g., "\begin{figure*}[!b]" is not normally possible in
% LaTeX2e). It also provides a command:
%\fnbelowfloat
% to enable the placement of footnotes below bottom floats (the standard
% LaTeX2e kernel puts them above bottom floats). This is an invasive package
% which rewrites many portions of the LaTeX2e float routines. It may not work
% with other packages that modify the LaTeX2e float routines. The latest
% version and documentation can be obtained at:
% http://www.ctan.org/pkg/stfloats
% Do not use the stfloats baselinefloat ability as the IEEE does not allow
% \baselineskip to stretch. Authors submitting work to the IEEE should note
% that the IEEE rarely uses double column equations and that authors should try
% to avoid such use. Do not be tempted to use the cuted.sty or midfloat.sty
% packages (also by Sigitas Tolusis) as the IEEE does not format its papers in
% such ways.
% Do not attempt to use stfloats with fixltx2e as they are incompatible.
% Instead, use Morten Hogholm'a dblfloatfix which combines the features
% of both fixltx2e and stfloats:
%
% \usepackage{dblfloatfix}
% The latest version can be found at:
% http://www.ctan.org/pkg/dblfloatfix




%\ifCLASSOPTIONcaptionsoff
%  \usepackage[nomarkers]{endfloat}
% \let\MYoriglatexcaption\caption
% \renewcommand{\caption}[2][\relax]{\MYoriglatexcaption[#2]{#2}}
%\fi
% endfloat.sty was written by James Darrell McCauley, Jeff Goldberg and 
% Axel Sommerfeldt. This package may be useful when used in conjunction with 
% IEEEtran.cls'  captionsoff option. Some IEEE journals/societies require that
% submissions have lists of figures/tables at the end of the paper and that
% figures/tables without any captions are placed on a page by themselves at
% the end of the document. If needed, the draftcls IEEEtran class option or
% \CLASSINPUTbaselinestretch interface can be used to increase the line
% spacing as well. Be sure and use the nomarkers option of endfloat to
% prevent endfloat from "marking" where the figures would have been placed
% in the text. The two hack lines of code above are a slight modification of
% that suggested by in the endfloat docs (section 8.4.1) to ensure that
% the full captions always appear in the list of figures/tables - even if
% the user used the short optional argument of \caption[]{}.
% IEEE papers do not typically make use of \caption[]'s optional argument,
% so this should not be an issue. A similar trick can be used to disable
% captions of packages such as subfig.sty that lack options to turn off
% the subcaptions:
% For subfig.sty:
% \let\MYorigsubfloat\subfloat
% \renewcommand{\subfloat}[2][\relax]{\MYorigsubfloat[]{#2}}
% However, the above trick will not work if both optional arguments of
% the \subfloat command are used. Furthermore, there needs to be a
% description of each subfigure *somewhere* and endfloat does not add
% subfigure captions to its list of figures. Thus, the best approach is to
% avoid the use of subfigure captions (many IEEE journals avoid them anyway)
% and instead reference/explain all the subfigures within the main caption.
% The latest version of endfloat.sty and its documentation can obtained at:
% http://www.ctan.org/pkg/endfloat
%
% The IEEEtran \ifCLASSOPTIONcaptionsoff conditional can also be used
% later in the document, say, to conditionally put the References on a 
% page by themselves.




% *** PDF, URL AND HYPERLINK PACKAGES ***
%
%\usepackage{url}
% url.sty was written by Donald Arseneau. It provides better support for
% handling and breaking URLs. url.sty is already installed on most LaTeX
% systems. The latest version and documentation can be obtained at:
% http://www.ctan.org/pkg/url
% Basically, \url{my_url_here}.




% *** Do not adjust lengths that control margins, column widths, etc. ***
% *** Do not use packages that alter fonts (such as pslatex).         ***
% There should be no need to do such things with IEEEtran.cls V1.6 and later.
% (Unless specifically asked to do so by the journal or conference you plan
% to submit to, of course. )

\usepackage[
  separate-uncertainty = true,
  multi-part-units = repeat
]{siunitx}

\usepackage{gensymb}
\usepackage{listings}
\usepackage{graphicx}
\usepackage{amsmath}

% correct bad hyphenation here
\hyphenation{op-tical net-works semi-conduc-tor}

\usepackage[utf8]{inputenc}
 
\usepackage{listings}
\usepackage{color}
\usepackage[american]{circuitikz}
 
\definecolor{codegreen}{rgb}{0,0.6,0}
\definecolor{codegray}{rgb}{0.5,0.5,0.5}
\definecolor{codepurple}{rgb}{0.58,0,0.82}
\definecolor{backcolour}{rgb}{0.95,0.95,0.92}
 
\lstdefinestyle{mystyle}{
    backgroundcolor=\color{backcolour},   
    commentstyle=\color{codegreen},
    keywordstyle=\color{magenta},
    numberstyle=\tiny\color{codegray},
    stringstyle=\color{codepurple},
    basicstyle=\footnotesize,
    breakatwhitespace=false,         
    breaklines=true,                 
    captionpos=b,                    
    keepspaces=true,                 
    numbers=left,                    
    numbersep=5pt,                  
    showspaces=false,                
    showstringspaces=false,
    showtabs=false,                  
    tabsize=2
}
 
\lstset{style=mystyle}


\begin{document}
%
% paper title
% Titles are generally capitalized except for words such as a, an, and, as,
% at, but, by, for, in, nor, of, on, or, the, to and up, which are usually
% not capitalized unless they are the first or last word of the title.
% Linebreaks \\ can be used within to get better formatting as desired.
% Do not put math or special symbols in the title.
\title{Parallel Matrix Transposition Using MPI With Derived Datatypes}
%
%
% author names and IEEE memberships
% note positions of commas and nonbreaking spaces ( ~ ) LaTeX will not break
% a structure at a ~ so this keeps an author's name from being broken across
% two lines.
% use \thanks{} to gain access to the first footnote area
% a separate \thanks must be used for each paragraph as LaTeX2e's \thanks
% was not built to handle multiple paragraphs
%

\author{\IEEEauthorblockN{\textbf{Sanele Ndlovu-716411 \\ Knowledge Dzumba - 813137} \\
		\IEEEauthorblockA{School of Electrical and Information Engineering, University of the Witwatersrand, Johannesburg \\
		ELEN4020A:Data Intensive Computing in Data Science}}}

% make the title area
\maketitle

% As a general rule, do not put math, special symbols or citations
% in the abstract or keywords.
\begin{abstract}
A parallel matrix transposition algorithm was partially implemented using MPI with derived datatypes. Data for the matrix was randomly generated by different processes and later combined by the master (rank 0) process to give the full matrix. The algorithm presented makes use of the MPI\_Type\_create\_subarray derived datatype to create subarrays of a larger matrix that are further transposed individually and then merged back together to form a transposed matrix. Due to difficulties encountered during the implementation phase, the transposed subarrays could not be merged into one output array.
\end{abstract}

% Note that keywords are not normally used for peerreview papers.
\begin{IEEEkeywords}
MPI, MPICH, Parallel transposition, collective I/O
\end{IEEEkeywords}



\section{Introduction}

\noindent
This paper presents the development and partial implementation of parallel matrix transposition using MPI derived datatypes and collective I/O. Performing operations on large datasets using traditional procedural methods of programming can have a significant impact on the efficiency and scalability of the program. With the shift in computer processor architecture and the current dominance of multi-core processors, parallel programming brings a solution to the problem of inefficient processing of large datasets. Parallel computing is divided into two categories, shared-memory computing and distributed memory computing. In shared memory computing, a collection of autonomous processors is connected to the main memory via an interconnection network and each processor has direct access to this main memory[1]. Shared memory computing facilitates communication between these processors through accessing shared data structures. Distributed memory computing comprises of multiple processors, which are each paired with their own private memory. The processors communicate with each other via an interconnection network, and this is usually done explicitly through message passing or through the use of functions that are specifically implemented to facilitate this type of communication. MPI (Message Passing Interface) is a standard for message passing in distributed memory systems, and it is mostly used with C, C++ and Fortran programming languages. Different implementations of this standard exist, the major ones being OpenMPI and MPICH. In this paper, the code was implemented using MPICH3.3 implementation of MPI. MPI derived data types are that standard's way of attempting to consolidate data that might otherwise require multiple messages so that they can be communicated between different processes. They are necessary because they make communication between processes cheaper since fewer messages get to be passed between processes. This is because in distributed memory systems, the cost of sending messages between processes is greater than the cost of performing computations in the local process[1].
\\

\noindent
Matrix transposition is one of the basic fundamental operations that can be performed on matrices, and they have a lot of applications in areas where data is to be processed and manipulated in a particular way, including having applications in many numeric algorithms such as FFT and the conversion of storage layout for arrays[2]. The focus of this paper is on parallel transposition of square matrices using MPI. The problem of a square matrix transposition can simply be explained as follows: Given a square $N\times N$ matrix, a new matrix has to be constructed such that row $i$ of the constructed matrix is similar to column $i$ of the original matrix. This means that the transposition operation exchanges the rows and columns of some matrix $M$. 



\section{Problem Description}

\noindent
It was required that a parallel matrix transposition algorithm be developed and implemented using MPI with derived datatypes. The matrix was to be made up of random short integers that are generated simultaneously by a collection of processes which could either be 16, 32 or 64. The matrices generated were to be of different sizes, specifically $2^N$ where $N$ is in the set ${3, 4 ... 7}$. The implementation was required to make use of collective I/O for reading and writing matrix data to files and it also had to make use of MPI's derived datatypes. Collective I/O is in this paper referred to as a type of communication in which all processes belonging to a particular MPI communicator are involved.

\section{Input Generation}

\noindent
To generate the contents for the matrices, the random number generator provided by the C-language standard library was used. Given the size of the matrix that was to be generate, all processes started by the program generated a sequence of $N\_local$ random short integers. In this case, $N\_local$ was chosen according to a simple partitioning scheme that divided the total number of elements the matrix is to store by the number of processes that have been started by the running program, that is 
\begin{equation}
    N\_local = \frac{N\timesN}{comm\_size}
\end{equation}

\noindent
After generating $N\_local$ random short integers, each process stored them in a local buffer of the same size. Each process then wrote the content of its buffer to a shared file, and no order of writing to the file was followed in this operation. The code that was used to achieve this can be seen in Listing 1.

\begin{lstlisting}[language = C, caption = input matrix file generation]

    //Each process should generate enough elements to fill a squre submatrix of main matrix
    local_N = (N * N / size);

    offset = rank * local_N * sizeof(int) + sizeof(int);

    int local_Buffer[local_N];

    for (int i = 0; i < local_N; i++)
    {
        local_Buffer[i] = rand() % 100;
    }

    //All processes write their randomy generated matrix elements to the same file
    MPI_File_write_at_all(fhw, offset, local_Buffer, local_N, MPI_INT, &status);
\end{lstlisting}

\noindent
Although the different processes wrote their randomly generated integers to the input matrix file, its content still had to be made available to the code for further processing of an actual 2d matrix. To do this, the master process (rank 0) was used to read the contents of the input matrix file into a global 2d array representing the matrix to be transposed. Listing 2 shows a snippet of the code that was used to achieve this part of the input matrix generation.


\begin{lstlisting}[language = C, caption = input matrix generation]

     //Creating the input matrix to be transposed. This is done by the master process
    if (rank == 0)
    {
        //Start reading actual matrix data at an offset of sizeof(int) to account for the
        //added matrix size
        MPI_File_read_at(fhw, sizeof(int), &inputMatrix[0][0], N, MPI_INT, MPI_STATUS_IGNORE);

        for (int i = 1; i < size; i++)
        {
            //Fill the global matrix from elements in the inputMatrixFile
            MPI_File_read_at(fhw, i * sizeof(int) * local_N + sizeof(int), &inputMatrix[i][0], N, MPI_INT,
                             &status);
        }
\end{lstlisting}


\section{Transposition Algorithm}

\noindent
The input matrix that was previously generated was chunked using MPI's derived datatype for generating subarrays. This was done so that a block transposition algorithm could be implemented, in which chunks of the original matrix are transposed separately and the overall transposed matrix acquired through combining these transposed chunks together. Although we failed to implement the combination of transposed chunks into an overall transposed matrix, an algorithm of how this was implemented as well as an argument for its correctness is discussed in this section. Given the sample matrix in Fig. 1. to be the matrix to be transposed. Suppose 16 processes are to be used in the transposition of this matrix. Then the matrix can be chunked into 16 smaller $2\times2$, examples of which are shown in blue and brown. Each of these smaller chunks are transposed independently by the processes that have been assigned to them. The assignment of processes to different subarray is performed by the master process, which also participates in the transposition of its self-assigned subarray. Given that the block length for the subarrays is 2 in this example, the master process assigns the block starting at index $[0;1\times blockSize]$ to the successor process 1, followed by the assignment of the block starting at $[0;2\times blockSize]$ to process number 2. This process of assignment continues in this order as long as $i\times blockSize$ does not exceed the length of one dimension of the original matrix to be transposed ($N$). In the event that $i\times blockSize$ exceeds the length of the matrix size $N$, the assignment is moved to the block starting at index $[blockSize; 0]$ and so on until all the blocks of the same size have been assigned to a process in the communicator. Once the assignment of blocks has been done, each process performs a transpose operation to the smaller chuck that it has been assigned to by applying a trivial transpose function as shown in Listing 3.

\newcommand*{\xMin}{0}%
\newcommand*{\xMax}{8}%
\newcommand*{\yMin}{0}%
\newcommand*{\yMax}{8}%
\begin{figure}
\begin{tikzpicture}
    \foreach \i in {\xMin,...,\xMax} {
        \draw [very thin,gray] (\i,\yMin) -- (\i,\yMax)  node [below] at (\i,\yMin) {$\i$};
    }
    \foreach \i in {\yMin,...,\yMax} {
        \draw [very thin,gray] (\xMin,\i) -- (\xMax,\i) node [left] at (\xMin,\i) {$\i$};
    }

\draw [step=1.0,blue, very thick] (0,0) grid (2,2);
\draw [very thick, brown, step=1.0cm,xshift=6cm, yshift=6cm] (0,0) grid +(2,2);
\end{tikzpicture}
\caption{Sample $8\times8$ matrix}
\end{figure}


\begin{lstlisting}[language = C, caption = transpose computation function]
int **transposeMatrix(int **matrix, int n)
{
    int **transposed = allocArray(n);
    for (int i = 0; i < n; i++)
    {
        for (int j = 0; j < n; j++)
        {
            transposed[i][j] = matrix[j][i];
        }
    }

    return transposed;
}
\end{lstlisting}

\noindent
Once the transposition of the chunks is complete, they have to be merged back into a complete matrix that is itself a transposed version of the original matrix. This requires a second transpose, but this time, instead of the numbers within the chunks being interchanged, the subarrays are the ones that get interchanged. This means that once process 0 is done transposing the blue subarray shown in Fig. 1. and process 12 is done transposing the brown subarray, these two subarrays are then interchanged such that process 0 now handles the transposed brown subarray while process 12 handles the transposed blue subarray.

\subsection{Programming Environment}

\noindent
The implemented code was compiled using GCC-7.0.4 on a Lenovo Ideapad 320 laptop with a RAM of 4GB and a processor speed of 2.5 GHz. MPICH3.3 was used to execute the program. Although MPI is a standard for distributed memory programming and allows for the use of different nodes to run the tasks, only one laptop was used to test and run the program implemented. Since no other

\section{Observation and Recommendations}

\noindent
It was observed that using the $rand()$ and the current time for random number generation for parallel programming is not ideal because in the generation of the input array, most processes generated the same numbers because they were being run exactly at the same time, hence the random number generator seed kept producing the same results. Difficulties that led to the combination of chunks of matrices back into a fully transposed matrix were identified to be due to the memory layout which was chosen for representation of arrays and subarrays. They made it difficult to combine the chunks back together into one array, hence it would be recommended that for future work, better care be taken in the development stage of the algorithm so that problems like this can be avoided.

\section{Conclusion}

\noindent
Worked done on the development and implementation of a parallel algorithm for matrix transposition has been presented. The algorithm is a block transposition algorithm in which a process is assigned a smaller chunk of the main matrix to be transposed and performs the transpose operation on this smaller chunk. On completion of this transposition, the blocks that have been assigned to different processes are again transposed using the same idea of interchanging rows and columns, but this time around, the chunks of transposed data and not the individual elements within the subarrays. The implementation of combining back the chunks into a complete matrix was not implemented, but an overview of how the algorithm was intended to operate has been described. The use of $rand()$ together with $srand(time(NULL))$ has been found to be non-ideal for programming parallel systems because one ends up with the same numbers generated by most of the processes involved.

% use section* for acknowledgment
%\section*{Acknowledgment}

% references section

% can use a bibliography generated by BibTeX as a .bbl file
% BibTeX documentation can be easily obtained at:
% http://mirror.ctan.org/biblio/bibtex/contrib/doc/
% The IEEEtran BibTeX style support page is at:
% http://www.michaelshell.org/tex/ieeetran/bibtex/
%\bibliographystyle{IEEEtran}
% argument is your BibTeX string definitions and bibliography database(s)
%\bibliography{IEEEabrv,../bib/paper}
%
% <OR> manually copy in the resultant .bbl file
% set second argument of \begin to the number of references
% (used to reserve space for the reference number labels box)
\begin{thebibliography}{1}

\bibitem{IEEEhowto:kopka}
P Pacheco, \emph{An Introduction to parallel programming}. Morgan Kaufmann: Fourth Edition

\bibitem{IEEEhowto:kopka}
J Gomez-Luna, I-J Sung, L Chang, J M Gonzalez, N Guil, W. W Hwu, \emph{In-Place Matrix Transposition on GPUs}, IEEE Transactions on Parallel and Distributed Systems, vol.27, No.3, March 2016

\bibitem{IEEEhowto:kopka}
https://www.merriam-webster.com/dictionary/risetime Accessed 12 April 2019
\end{document}


