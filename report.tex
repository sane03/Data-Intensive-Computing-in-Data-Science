\documentclass[12pt]{article}
\usepackage{fullpage}
\usepackage{times}
\usepackage[normalem]{ulem}
\usepackage{fancyhdr,graphicx,amsmath,amssymb, mathtools, scrextend, titlesec, enumitem}
\usepackage[ruled,linesnumbered]{algorithm2e} 
\include{pythonlisting}

\title{Lab 1: 3-D by 3-D Array Multiplication}
\author{Knowledge Dzumba (813137), Sanele Ndlovu (716411)}
\begin{document}
\maketitle

\pagebreak

\noindent
\textbf{Algorithm 1} computes the addition of two matrices \emph{A} and \emph{B}. This is done by adding corresponding elements in both these matrices. The result is returned in matrix \emph{C}

\begin{algorithm}
\caption{rank2TensorAdd}
\KwIn{\emph{n} x \emph{n} matrices \emph{A} and \emph{B}}
\KwOut{\emph{n} x \emph{n} matrix \emph{C}}
\hrulefill\\

\emph{n = A.rows} \\
\emph{Let C be a newly created} \emph{n} x \emph{n} \emph{matrix} \\

\nl \For{i = 0 to n - 1} {
	\nl \For{j = 0 to n - 1}{
    	\nl \emph{C[i][j] = A[i][j] $+$ B[i][j]}
    	}
        }
\nl \Return{$C$}
\end{algorithm}

\noindent
\textbf{Algorithm 2} computes the multiplication of two matrices \emph{A} and \emph{B}. The algorithm uses rank 2 tensor contraction method to compute this multiplication and the result is returned in \emph{C}

\begin{algorithm}
\caption{rank2TensorMult}
\KwIn{\emph{n} x \emph{n} matrices \emph{A} and \emph{B}}
\KwOut{\emph{n} x \emph{n} matrix \emph{C}}
\hrulefill\\

\emph{n = A.rows} \\
\emph{Let C be a newly created} \emph{n} x \emph{n} \emph{matrix} \\
\nl \For{i = 0 to n - 1} {
	\nl \For{j = 0 to n - 1}{
	    \nl \emph{C[i][j] = 0}
	    
	    \For{k = 0 to n - 1}
	    {
	        \emph{C[i][j] = C[i][j] $+$ A[i][k]$*$B[k][j]}
	    }
    	}
    }
\nl \Return{$C$}
\end{algorithm}

\noindent
\textbf{Algorithm 3} performs 3D matrix addition of two 3D matrices \emph{A} and \emph{B}. This operation is an element wise addition, so corresponding elements in matrix \emph{A} and matrix \emph{B} are simply added together and the result of the addition is returned in matrix \emph{C}

\pagebreak


\begin{algorithm}
\caption{rank3TensorAdd}
\KwIn{\emph{n} x \emph{n} x \emph{n} matrices \emph{A} and \emph{B}}
\KwOut{\emph{n} x \emph{n} x \emph{n} matrix \emph{C}}
\hrulefill \\

\emph{n = A.rows} \\
\emph{Let C be a newly created} \emph{n} x \emph{n} x \emph{n} \emph{matrix} \\
\nl \For{i = 0 to n - 1} {
    \nl \For{j = 0 to n - 1}{
        \nl \For{k = 0 to n - 1}
        {
            \emph{C[i][j][k] = A[i][j][k] $+$ B[i][j][k]}
        }
    }
}
\nl \Return{$C$}
\end{algorithm}

\noindent
\textbf{Algorithm 4} performs 3D matrix multiplication of two 3D matrices \emph{A} and \emph{B}. This is done using a similar strategy as that used for 2D matrix multiplication, modified to fit the case of a 3D matrix. In this algorithm, a row is considered to be a 2D horizontal matrix that is part of the 3D matrix (cube) \emph{A}, and a column is considered to be a left-facing 2D vertical matrix that is also part of a 3D matrix \emph{B}. In the case of 2D matrix multiplication, a row of some matrix \emph{D} is multiplied by a column of some second matrix \emph{D} to produce a single element(scalar) of the resulting matrix \emph{F}, while in the case of 3D matrix multiplication, a row of matrix \emph{A} is multiplied by a column of matrix \emph{B} to produce a vector whose elements form part of the resulting \emph{C} matrix. An element at index \emph{i,j,k} of the resulting \emph{C} matrix is found using the formula:

\begin{align*}
  C_{ijk} &= \sum_{l} A_{ilk}\times B_{ljk} \\
\end{align*}\begin{align*}

\noindent It is assumed in this algorithm that the dimensions of both \emph{A} and \emph{B} matrices are the same, that is they are both \emph{n} \times \emph{n} \times \emph{n}.

\pagebreak

\begin{algorithm}
\caption{rank3TensorMult}
\KwIn{\emph{n} x \emph{n} \emph{n} matrices \emph{A} and \emph{B}}
\KwOut{\emph{n} x \emph{n} x \emph{n} matrix \emph{C}}
\hrulefill \\

\emph{n = A.rows} \\
\emph{Let C be a newly created} \emph{n} x \emph{n} x \emph{n} \emph{matrix} \\

\nl \For{i = 0 to n - 1}{
    \nl \For{j = 0 to n - 1}{
        \nl \For{k = 0 to n - 1}
        {
            \emph{C[i][j][k] = 0}
            
            \nl \For{l = 0 to n - 1}
            {
                \emph{C[i][j][k] = C[i][j][k] $+$ A[i][l][k]*B[l][k][k]}
            }
        }
    }
}
\nl \Return{$C$}

\end{algorithm}
\end{document}